\chapter{MEAN Frameworks (Fabian)}\label{mean-frameworks-fabian}

\section{Mean.io}\label{mean.io}

Mean.io ist ein Framework, dass auf die Kombination der verbreiteten
Technologien MongoDB, Node.js, Angular.js und Express.js setzt. Es wurde
von dem Freelancer Amos Haviv in Kooperation mit dem Unternehmen
Linnovate als Open Source veröffentlicht. Weiterhin sollte Amos Haviv
dieses für die Community pflegen. Es zeichnet sich dadurch aus, dass es
eines der ersten Frameworks ist, die auf dem MEAN Stack basieren. Nicht
zuletzt dadurch hat Mean.io eine beachtliche Community.

\section{Mean.js}\label{mean.js}

Durch einen Interessenskonflikt zwischen Amos Haviv und dem Unternehmen
Linnovate entschied sich dieser ein neues Framework namens Mean.js
aufzubauen und sich von Linnovate zu trennen. Dieses entstand im Februar
2014 aus einem fork von Mean.io. Da Mean.io Open Source ist war dieses
Vorgehen rechtlich möglich. Das Ziel ist die Entwicklung eines produktiv
einsetzbaren Frameworks. Obwohl alle Technologien von MEAN einzeln
betrachtet produktiv einsetzbar sind, erweist sich die Kombination
dieser als problematisch.

Die Entwickler haben aus den Problemen von Mean.io, zum Beispiel dem
nicht vorhandenen Versions Schema gelernt. Was wurde gegenüber Mean.io
geändert?

Modularity - Support von MVC Modulen - Angular.js, Support vertikaler
Module

Legacy Support - Versions Schema eingeführt

Community - Besserer Support für MEAN Entwickler - Twitter, Facebook,
Google Gruppe, IRC-Channel (real-time support)

Zukunftsziele - Mean Kern verbessern und Fehler beheben (Bug fixing) -
Building companion modules to extend MEAN with different web application
features - Building a Yeoman Generator - Admin panel for managing MEAN
application

\section{Meteor.js}\label{meteor.js}

Ein weiteres interessantes Open-Source Framework ist Meteor.js. Die
Initiale Veröffentlichung dieses Frameworks war 2012. Es ist für die
Entwicklung von Echtzeit Web Anwendungen gedacht. Es handelt sich
hierbei um eine client-server Plattform. Meteor.js setzt auf die
Technologien MongoDB, Node.js und Angular.js. Die Aktuelle Version ist
die 1.2.1. Interessant zu erwähnen ist das bereits 2012 11.2 Millionen
\$ von Sponsoren zur Verfügung gestellt wurden.

Vorteile - Schnell \& einfach zu lernen / entwickeln - Gute
Dokumentation - Bietet viel ``out of the box'' - Keine callback hell

Nachteile - frühes Entwicklungsstadium - Automatisiertes Testen wird
nicht unterstützt

Zukunftsaussichten

Es soll nicht nur MongoDB unterstützt werden, sondern noch viele andere
Datenbanken. Unter anderem auch SQL-Datenbanken.

Wird von manchen als möglicher Nachfolger für MEAN.io angesehen.
