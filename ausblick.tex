\chapter{Ausblick (Marinus)}\label{ausblick-marinus}

Welche Änderungen sind in naher Zukunft zu erwarten?

Während der Entwicklung der Fallstudie FairProjects, wurde im Dezember
2015 die Version 2 von Angular vom Alpha zum Beta Status geändert. Es
ist deshalb zu erwarten dass in naher Zukunft die neue Angular Version
offizell veröffentlicht wird. Gerade in der Schnelllebigkeit der
JavaScript Entwicklung, ist es fraglich, wie lange Angular der Version 1
dann noch aktiv unterstützt wird. Auch wenn zu erwarten ist, dass es
Hilfestellungen zum Upgrade geben wird, ist ein Umstieg auf die neue
Version unweigerlich mit Mehraufwand verbunden. Da sich nicht nur die
Schnittstellen sondern auch zentrale Konzepte verändert haben, muss mit
zusätzlicher Einarbeitungszeit für Entwickler gerechnet werden.

Führt man diesen Gedanken weiter, ist es mehr als fraglich, heute noch
Entwickler für Angular 1 weiterzubilden. Wenn das Know-How nicht bereits
vorhanden ist, kann dieser Schritt nicht mit gutem Gewissem empfohlen
werden. Da Angular 2 jedoch aktuell noch nicht über den Beta-Status
hinaus ist, steht man an dieser Stelle vor einer schwierigen Situation.
Auch wenn die grundlegenden Änderungen vorgenommen wurden, sind
Änderungen an der API noch möglich beziehungsweise noch zu erwarten.

Wenngleich der Versionswechsel von Angular aktuell den Einsatz des
Frameworks erschwert, ist er unserer Meinung nach der größte
Hoffnungsschimmer. Nach einer längeren experimentellen Phase setzt
Angular in der neuen Version offiziell auf die Sprache TypeScript.

\emph{TypeScript} ist eine von Microsoft entwickelte Sprache die
JavaScript erweitert. Gülitiger JavaScript Code kann dabei auch in
TypeScript ausgeführt werden. Zusätzlich werden jedoch auch
Sprachkonstrukte wie Klassen, Interfaces und Generics unterstützt.
Grundlegender unterschied ist auch die statische Typisierung von
TypeScript. Mit Compilern kann TypeScript Code in gültigen JavaScript
Code nach Standard ECMAScript 5 kompiliert werden. Einige der Konzepte
von TypeScript sollen in den Standard ECMAScript 6 einfließen.

Für die Zukunft könnte der MEAN Stack komplett mit TypeScript umgesetzt
werden. Der ursprüngliche Gedanke eine Sprache für alle Schichten einer
Webanwendung zu haben bleibt dadurch erhalten. Durch die zusätzlichen
Sprachkonzepte fallen jedoch viele Einschränkungen basierend auf der
dynamischen Typisierung weg. Das Entwickeln dürfte dadurch erleichtert
werden. Durch Konzepte wie Klassen und Vererbung, wird auch der Einstieg
für JavaScript-unerfahrene Entwickler leichter als in die
Prototyp-basierte Vererbung in JavaScript fallen.
