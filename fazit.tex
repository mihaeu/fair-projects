\chapter{Fazit (Marinus)}
\label{fazit}

Ist die Entwicklung von Webanwendungen basierend auf dem MEAN-Stack zu
empfehlen?

Unserer Meinung nach: \emph{größtenteils nein}. Es mag Einsatzszenarien
geben bei denen dieser Ansatz seine Vorteile hat. Bei vielen Projekten
dürfte es jedoch an den zu erwarteten Wartungskosten scheitern. Bereits
in der kurzen Entwicklungszeit für diese Fallstudie hatten wir mehrmals
mit den Auswirkungen von Abhängigkeitskonflikten, als auch nicht mehr
weiterentwickelten Fremdbibliotheken, zu kämpfen. Da Express nur sehr
wenig Unterstützung für Standardproblemstellungen liefert, ist mit
gleichen Problemen bei fast jedem Projekt zu rechnen. Was bei
Änderungswünschen an unserer Fallstudie nötig wäre, kann momentan nicht
abgeschätzt werden. Es ist für uns jedoch nicht abwegig dass einige der
von uns aktuell verwendeten Pakete in der jetzigen Form nicht mehr
unterstützt werden. Uns ist bewusst dass dieses Problem auch mit anderen
Technologien auftritt, die kurze Zeitspanne, in der oftmals weit
eingesetzte Pakete nicht abwärtskompatible Änderungen vollziehen, lässt
jedoch aufhorchen.

Eine Ausnahme darf aus unserer Sichtweise gemacht werden:
\emph{Prototyping}. Um Prototypen für Beispielsweise
Marktbedarfsanalysen erstellen zu können, eignet sich der MEAN Stack
bestens. Die oftmals beschränkten Ressourcen sprechen für das Szenario
eine Sprache für alle Technologien zu setzen. Des Weiteren kann durch
den schemalosen Ansatz von MongoDB einfach auf Änderungen am Datenmodell
reagiert werden ohne bisherige Datensätze aufwendig migrieren zu müssen.
Erst dies ermöglicht einen iterativen Entwicklungsprozess.

Zweifelsohne haben alle Technologien für sich selbst ihre
Daseinsberechtigung. Wir würden sie jederzeit wiederverwenden. Meist
jedoch nur einzelne Teile, angepasst auf das jeweilige Problem. Die
Anforderungen die heute an eine Webanwendung gestellt werden, sind so
unterschiedlich, als dass sie alle mit einer Patentlösung abgedeckt
werden könnten.
