\section{Fallstudie: Fair Projects (Markus)}
\label{fallstudie-fair-projects}

Als Beispielanwendung für die Erstellung einer MEAN Anwendung kommt das Problem der Verteilung von Projekten zum Einsatz.
So sind Projektarbeiten im Studienumfeld oft anzutreffen und die zu erarbeitenden Themen sind dabei für die Teilnehmer oftmals vordefiniert.
Jetzt stellt sich das Problem, welcher Teilnehmer, welches Projekt zugeteilt bekommen.
Die Verteilung ist dabei oft recht langwierig.
Um dieses Problem zu lösen ist das Windhundprinzip sehr beliebt. Dort erhält der der zuerst kommt, den Zuschlag erhält.
Das fürht meist dazu, dass dies der mit dem am stärksten ausgeprägten Stimmorgan ist. 

Unsere Idee, um dieses Problem fair zu lösen, ist die Verteilung mittels Priorisierung.
Dazu soll eine Webanwendung entwickelt werden, auf die alle Teilnehmer Zugriff haben. 
Dort soll jeder Teilnehmer seine Lieblingsprojekte je nach Wunsch individuell priorisieren können.
Der Ersteller der Projekte soll dabei immer einen Überblick haben, welche Projekte unterbelegt sind und wie die einzelnen Prioritäten der Teilnehmer verteilt sind.
Dadurch soll die aufgewendete Zeit verringert werden und eine möglichst faire Wahl für alle Teilnehmer gewährleistet werden.

Der grundlegende Ablauf dabei sieht folgendermaßen aus:
\begin{itemize}
	\item Die Professoren erstellen für das jeweilige Fach mehrere Projekte mit vollständiger Beschreibung
	\item Die Studenten können anschließend ihre Prioritäten für jedes der Projekte vergeben
	\item Der Professor kann die Projekte anschließend, anhand der Übersichtstabelle mit allen Prioritäten, verteilen
\end{itemize}
\begin{itemize}
	\item Mockups
	\item Architektur
	\item REST API (bitte kurz definieren, damit ich mich darauf beziehen kann)
	\item Erklärung der Funktionen
\end{itemize}