\chapter{AngularJS (Markus)}
\label{angularjs}

AngularJS ist eines der bekanntesten JavaScript Frameworks welches 2009 von Google veröffentlicht wurde. 
Es stellt einen Aufsatz zu reinen HTML-Seiten dar. 
Klassische Websites setzen meist auf PHP im Backend, welcher bei jedem Aufruf ein einzelne HTML Webseite generiert.
Beim Durchführen einer Aktion wie beispielsweise dem Aufrufen einer Detailansicht wird ein neuer Request an das PHP Backend durchgeführt und eine komplett neue HTML Webseite zurück geliefert.
Die Webanwendung ist dabei ausgelegt auf Thin Clients.
Der HTML Part davon ist dazu meist passiv und dient nur zur Anzeige.
Im MEAN Stack übernimmt AngularJS dabei die Aufgabe mehr Logik aus dem Backend in die View zu übernehmen.
Daraus folgt auch, dass es mehr auf Fat Clients ausgelegt ist die viele Berechnungen und Abläufe erledigen ohne den Anwendungsstatus bei jeder Aktion neu generieren zu lassen.
Dieser generelle Unterschied zu den klassischen Websites wird unterschieden durch den Begriff "stateful".
Der View ist im Gegensatz zu stateless der komplette Status der Webanwendung bekannt.
Die Daten werden dabei im Browser zwischengespeichert.
Zur Synchronisation mit dem Server werden nur die geänderten Daten in beide Richtungen ausgetauscht.
Ein weiteres Grundprinzip von Angular ist es sogenannte Single-Page-Anwendungen zu machen.
Bei dem Wechseln einer Seite zu beispielsweise einer Detailansicht wird kein neuer HTTP-Request verschickt, der die Seite austauscht.
Manche Daten sind der Webanwendung bereits bekannt.
Ist die Ansicht zum Beispiel die Übersicht aller Projekte aus dem Fallbeispiel, werden durch einen Klick auf ein einzelnes Projekt nur die noch fehlenden Daten angefordert.
In der Listenansicht ist der Titel bereits vorhanden, daher wird nur noch die Beschreibung und die Teilnehmer vom Backend benötigt.
Der komplette Rest der Seite kann dabei unverändert bestehen bleiben.
Dadurch wird bei jedem Request viel Overhead eingespart.

Um trotz Single-Page-Anwendung auf bestimmte Teile einer Anwendung mittels URL zugreifen zu können ist ein Routing nötig.
Wie in dem Beispiel TODO zu sehen, wird für jeden einzelnen Pfad der URL angegeben welches Template verwendet werden soll und welcher Controller.
Mit der .otherwise Funktion kann ein sogenannter Fallback implementiert werden für einen Aufruf der Hauptseite.
Innerhalb der URL können auch gleich Parameter mit Hilfe von vorangestellten Doppelpunkten angegeben werden die automatisch die erfolgte Eingabe in der URL im entsprechenden Controller bereitstellen.

Zur Darstellung der Daten setzt Angular auf das Model View ViewModel Prinzip (MVVM), das von Microsoft entwickelt wurde.
Für diesen Einsatzweck ist es passender als beispielsweise Model View Controller (MVC), da das ViewModel ein two-way data-binding zwischen der View und dem Model ermöglicht.
Anders als viele zuerst vermuten, stellt das ViewModel dabei keinen Ersatz für den Controller dar.
Dieser ist zwar nicht explizit in der Abkürzung genannt, wird aber weiterhin benötigt.
Im Unterschied zu MVVM werden in MVC bei Änderungen im View nicht automatisch im Model aktualisiert.
Das ViewModel hat daher die Aufgabe jede Änderung in der View auch an das Model zu übertragen und auch umgekehrt in die andere Richtung.
Wird beispielsweise das Model geändert wird diese Änderung sofort an die View übertragen.
Eines der Alleinstellungsmerkmale von AngularJS unter den JavaScript Frameworks stellt Dependency Injection dar.
Es ist ein Entwurfsmuster mit dem Abhängigkeiten zentral aufgelöst werden können.
Die benötigten Objekte werden zur Laufzeit zur Verfügung gestellt.
Dadurch wird die Austauschbarkeit erhöht und auch das Testen der Anwendung in dem Teile durch Mock-Objekte ausgetauscht werden.
Näheres dazu findet sich im Kapitel zur Testbarkeit.

AngularJS sieht für die Strukturierung folgendes vor:
\begin{itemize}
	\item Module
	\item Models
	\item View-Templates
	\item Controller
	\item Scopes
	\item Filter
	\item Services
\end{itemize}
Diese Komponenten können mittels eines Dependency Containers zusammengeführt werden.
Dadurch wird ein modulares Design ermöglicht welches Komponenten austauschbar macht.
Module:
Alle zusammengehörenden Komponenten einer Anwendung werden zur besseren Trennung in Module strukturiert.
Auch hier kommt das Konzept von Dependency Injection zum Tragen.
In einem Modul können wiederum weitere Module eingebunden werden.
Models:
AngularJS gibt für Models keine besonderen Klassen vor von denen geerbt werden müsste.
Diese stellen dadurch klassische JavaScript Objekte dar, die einen Aspekt der Domäne repräsentieren.
View-Templates:
Das sind in der Regel Teile von HTML Seiten die mit Platzhaltern oder bestimmten Angular-Direktiven gefüllt sind.
Beispielsweise der Platzhalter {{myVariable}} wird im HTML Quelltext durch die Variable aus dem Controller ausgetauscht.
Direktiven erlauben die Erweiterung des HTML um Tags.
Dabei gibt es viele nützliche Funktionen wie beispielsweise Objekte in einer Schleife durch zu iterieren. 
Controller:
In AngularJS werden Controller hauptsächlich zur Erstellung von Scopes verwendet.
Die Controller fallen dadurch auch recht kompakt aus.
Das heißt für jede CRUD Operation (Create Read Update Delete) und für jedes Model wird in AngularJS ein eigener Controller erstellt.
Beispielsweise ein ShowProjectsController.
Oftmals werden die Controller auch mit Ctrl abgekürzt.
In unserem Fallbeispiel haben wir uns aber zugunsten der besseren Verständlichkeit dagegen entschieden.
Jede komplexere Logik sollte in eigenen Services ausgelagert werden und wieder mittels Dependency Injection eingebunden werden. 
Scopes:
Scopes bündeln die Daten und Funktionen eines Controllers und ermöglichen die Kommunikation zwischen Controllern untereinander.
Sie beziehen sich immer nur auf einen bestimmten Abschnitt innerhalb des DOM (Document Object Model)
Filter: Haben wir glaub ich nicht wirklich verwendet? TODO
Services:
In unserer Beispielanwendung haben wir beispielsweise den \$resource Service eingebunden.
Damit wird ermöglicht, dass unsere REST Schnittstelle angesprochen werden kann.
Services werden generell als Singleton instanziiert.

Zur Selektierung von Elementen verwendet AngularJS ein integriertes jQuery Lite (jqLite).
Falls jQuery ebenfalls ausgeliefert wird, wird auch diese jQuery Version von AngularJS verwendet und nicht jqLite.

Aktuell befindet sich AngularJS in der Version 1.4.8.
Es ist jedoch schon seit längerer Zeit bekannt, dass die nächste große Version 2.0 nicht abwärtskompatibel sein wird mit Version 1.x.
Version 2.0 befindet sich im Moment noch im Beta Stadium und es ist noch nicht abzusehen wann es veröffentlicht wird.
Aus diesem Grund haben wir uns auch dagegen entschieden auf Version 2.0 zu setzen.
Je nach geplanter Lebensdauer der Webanwendung sollte entschieden werden ob ein Einsatz in Frage kommt.
