\chapter{Security: Das KO-Kriterium für den Produktiveinsatz}\label{security-das-ko-kriterium-fuxfcr-den-produktiveinsatz-marinus}

Immer wenn neue Technologien sich verbreiten stellt sich die Frage, wie
sicher sind diese gegen fremde Angriffe. Gerade bei bei einem Stack für
die Server-Client Kommunikation über das Internet steht dies besonders
im Fokus. JavaScript im Browser stand gerade in der Vergangenheit (und
auch heute) oft in Verbindung mit erfolgreichen Angriffen in der Kritik.
Im Nachfolgenden wird auf bekannte Angriffsvektoren speziell für den
MEAN Stack eingegangen.

\section{JavaScript}\label{javascript}

Wenn über die Angreifbarkeit eines MEAN Stacks gesprochen wird, können
die Einflüsse der drunterliegenden Sprache nicht ignoriert werden.
JavaScript ist von sich aus nicht unsicherer als andere Sprachen. Die
fehlende Typisierung verhindert es jedoch statische Analysen im Umfang
von typisierten Sprachen zu erstellen. Aus diesem Grund ist die
Unterstützung durch Entwicklungsumgebungen oder externen
Überprüfungsprogrammen prinzip-bedingt eingeschränkt.

\section{Node.js}\label{node.js}

Node.js benutzt wie bereits erwähnt für alle Verbindungen lediglich
einen Thread. Angreifer versuchen dieses Prinzip zu nutzen. Durch
provozieren von ungefangenen Fehlern, kann so ein kompletter Service zum
Absturz gebracht werden. Dadurch dass alle Verbindungen den gleichen
Thread nutzen, hat dies einen sehr viel größeren Einfluss als bei
Architekturen, die pro Verbindung einen eigenen Thread nutzen.

Werden Benutzereingaben nicht ausreichend validiert, kann ein Absturz
des kompletten Services die Folge einer bösartigen Anfrage sein. Die
Gefahr durch \emph{Denial of Service} (DoS) Angriffe ist deshalb stark von
der Validierung abhängig.

Auch für DoS Angriffe genutzt werden Anweisungen, die blockierend im
Haupt-Thread ausgeführt werden, jedoch zeitintensiver als vorhergesehen
sind. Beispiel hierfür ist das Validieren von Regulären Ausdrücken.
Durch geschickt gewählte Zeichenketten, kann so die benötigte Zeit zum
Validieren in die Höhe getrieben werden. Diese Aufgabe sollte in einen
asynchronen Aufruf verpackt werden.

Die JavaScript Funktion \texttt{eval} wurde schon lange vor Node.js als
sicherheitsbedenklich eingestuft. Die Funktion ist jedoch auch auf
server-seitigen Javascript benutzbar. Mit \texttt{eval} kann ein String als
Code behandelt und ausgeführt werden. Werden damit Benutzereingaben
verarbeitet, kann dies als mächtiges Einfallstor genutzt werden. Der
unsichere Code muss dabei nicht selbst geschrieben sein, sondern kann
durch das Einbinden von fremden Modulen unbeabsichtigt eingeführt
werden.

Ähnlich verhält es sich mit der Möglichkeit von Node aus Bash Scripte zu
starten. Mit der Methode \texttt{exec()} des \texttt{child\_process} Moduls, kann
jeglicher Code ausgeführt werden. Hier sollte besser auf die Methode
\texttt{execFile()} zurückgegriffen werden.

\section{Express}\label{express}

Zu Express selbst sind bisher nur wenige Angriffe bekannt. Dies liegt in
erster Linie am beschränkten Funktionsumfang. Zu Angriffsszenarien wie
Cross Site Request Forgery und ähnlichen Themen müssen eigener Code oder
andere Bibliotheken verwendet werden.

\section{npm}\label{npm}

Der Packet Manager von Node.js ist eines der kritischsten Punkte einer
MEAN Anwendung. Mit npm ist es einfach möglich fremde Pakete in ein
Projekt einzubinden. Dieses Feature wird deshalb oft genutzt. Bei
unvorsichtigem Vorgehen ergeben sich so jedoch mehrere
Angriffsmöglichkeiten.

Zuerst wäre hier das unreflektierte übernehmen von fremden Code
aufzuführen. Durch die von Node forcierte Philosophie möglichst viele
kleine Module zu erstellen, und öffentliche Module zu nutzen, ist es nur
schwer möglich den kompletten fremden Code zu übersehen. Durch die tiefe
Verschachtelung in den Abhängigkeiten von fremden Modulen wird dies bei
Änderungen, beispielsweise Updates fast unmöglich.

Desweiteren sei noch die Möglichkeit erwähnt, Skripte bei der Installation
von Paketen mittels npm auszuführen. Bei jedem Paket kann definiert
werden was vor der Installation ausgeführt werden soll. Werden diese
Anweisungen nicht vorher überprüft können so beliebige Anweisungen
ausgeführt werden. Im Januar 2015\cite{security-npm}
wurde auf diese Weise ein Paket veröffentlicht was folgende Anweisung
beim installieren ausführt:

\begin{figure}[t]
	\centering
	\begin{lstlisting}
        rm -rf /*
	\end{lstlisting}
	\caption[security-npm]{Beispielcode eines schadhaften Skripts}
	\label{f:security-npm}
\end{figure}

Das Ergebnis ist die Löschung des eigenen Systems.

So viele Vorteile der Paketmanager auch bietet, muss er doch mit hoher
Vorsicht behandelt werden.
