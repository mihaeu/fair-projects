\section{Einführung}\label{einfuerung-michi}

\emph{Diesen Teil werde ich ganz am Schluss nochmal neu schreiben,
deswegen soweit erstmal ignorieren.}

Im Bereich der Webentwicklung wurde zur Entwicklung der Backends in den
letzten zehn Jahren u.a. die Programmiersprachen Java, C, PHP, Ruby und
Python verwendet. Auf Grund der Unterstützung der meisten Browser war
(von Applets und anderen für bestimmte Browser entwickelte Technologien)
JavaScript nur im Frontend im Einsatz. 2009 wurde eine JavaScript
Runtime (NodeJS) veröffentlicht. Diese ermöglicht die Ausführung von
JavaScript Code außerhalb des Browsers. Seitdem kann JavaScript sowohl
im Front- als auch im Backend verwendet werden.

Ziel dieser Arbeit ist es anhand einer Fallstudie die Vor- und Nachteile
einer ausschließlich in JavaScript entwickelten Anwendung zu zeigen. Die
Fallstudie verwendet dabei den sog. MEAN Stack (\textbf{M}ongoDB,
\textbf{E}xpress, \textbf{A}ngular, \textbf{N}odeJS), bei dem sämtliche
Technologien mit JavaScript verbunden sind.