\section{Definition (Fabi)}
\label{definition-fabi}

Zuerst ist zu sagen, dass es sich bei MEAN nicht um einen fest definierten Stack handelt. MEAN bedeutet lediglich, dass die MongoDB, Express, AngularJS und NodeJS Technologien zum Einsatz kommen. Damit soll das MEAN Konzept den LAMP Stack ersetzen. Wenn man sich mit der MEAN Thematik beschäftigt stößt man daher auch auf mehrere unterschiedliche MEAN Stacks, unter anderem z.B. MEAN.IO oder MEAN.JS.

In der folgenden Abbildung sollen die Zusammenhänge der verwendeten Frameworks untereinander aufgezeigt werden.

TODO Abbildung

Bei MongoDB handelt es sich um die populärste NoSQL Datenbank, die vor allem für ihre Skalierbarkeit bekannt ist. Als Webframework für Node kommt das Express Framework zum Einsatz. Angular wird im Frontend zur Eintwicklung von Single-page-Webanwendungen nach dem MVC-Pattern eingesetzt und kommt aus dem Hause Google. Zu guter letzt wird das verbreitete Node Framework im Backend Bereich verwendet.

Alle vier Frameworks weisen eine große Verbreitung bzw. Community auf.

\emph{was ist MEAN}

\begin{itemize}
\itemsep1pt\parskip0pt\parsep0pt
\item
  ganz grobe Definition
\item
  Technologien im Detail spaeter
\item
  wer hat es gepraegt?
\item
  wer verwendet es? (das evtl. unter \hyperref[motivation]{Motivation})
\end{itemize}
