\chapter{Motivation}\label{motivation-markus}

Ein Großteil aller Webanwendungen ist in der Skriptsprache PHP geschrieben. 
So werden über 80\% aller Anwendungen damit geschrieben und ein verschwindend geringer Anteil von nur 0,1\% in JavaScript.
Dennoch stellt ein Grund sich näher mit dem MEAN Stack zu befassen die Tatsache dar, dass ein klarer Trend zu JavaScript erkennbar ist.
So ist das Interesse in den letzten Jahren enorm gestiegen wie auch auf TODO erkennbar ist.
Dort ist die Steigerung an Suchanfragen zu dem Thema NodeJS angegeben.
Im Vergleich dazu ist das Interesse an beispielsweise PHP entsprechend gesunken. 
Auch ist der Bekanntheitsgrad von den anderen Komponenten wie MongoDB oder AngularJS stark gestiegen.
Selbst große Firmen wie RedBull oder Netflix setzen Teile davon ein.
Oftmals wird das durch die verkürzte Entwicklungszeit zur Erstellung von Webanwendungen und durch die gesteigerte Performanz begründet.
Zudem hat es gewisse Vorteile für Unternehmen, dass es leichter fällt agile Methoden einzuführen.
Meist werden Aufgaben in die Teile Frontend und Backend aufgeteilt und von unterschiedlichen Personen erarbeitet.
Frontendentwicklern, die sich mit den Technologien HTML, CSS und JavaScript auskennen und Backendentwicklern, die nur die serverseitigen Komponenten mit beispielsweise PHP und MySQL bearbeiten.
Da im MEAN Stack, NodeJS, Express und AngularJS auf JavaScript basieren und sogar MongoDB JavaScript ausführen kann, können nun die Aufgaben anderes aufgeteilt werden.
So ist es nicht mehr nötig nach Frondend und Backend zu trennen sondern die Aufgaben vertikal zu schneiden, wodurch jeder Entwickler alle Bereiche eines Features bearbeiten kann.
Vgl. Elephant-Carpaccio von Alistair Cockburn. 
Grund genug sich näher mit den einzelnen Komponenten zu beschäftigen.

\chapter{Fallstudie: Fair Projects}\label{fallstudie-fair-projects}